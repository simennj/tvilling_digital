%% Generated by Sphinx.
\def\sphinxdocclass{report}
\documentclass[letterpaper,10pt,english]{sphinxmanual}
\ifdefined\pdfpxdimen
   \let\sphinxpxdimen\pdfpxdimen\else\newdimen\sphinxpxdimen
\fi \sphinxpxdimen=.75bp\relax

\PassOptionsToPackage{warn}{textcomp}
\usepackage[utf8]{inputenc}
\ifdefined\DeclareUnicodeCharacter
% support both utf8 and utf8x syntaxes
  \ifdefined\DeclareUnicodeCharacterAsOptional
    \def\sphinxDUC#1{\DeclareUnicodeCharacter{"#1}}
  \else
    \let\sphinxDUC\DeclareUnicodeCharacter
  \fi
  \sphinxDUC{00A0}{\nobreakspace}
  \sphinxDUC{2500}{\sphinxunichar{2500}}
  \sphinxDUC{2502}{\sphinxunichar{2502}}
  \sphinxDUC{2514}{\sphinxunichar{2514}}
  \sphinxDUC{251C}{\sphinxunichar{251C}}
  \sphinxDUC{2572}{\textbackslash}
\fi
\usepackage{cmap}
\usepackage[T1]{fontenc}
\usepackage{amsmath,amssymb,amstext}
\usepackage{babel}



\usepackage{times}
\expandafter\ifx\csname T@LGR\endcsname\relax
\else
% LGR was declared as font encoding
  \substitutefont{LGR}{\rmdefault}{cmr}
  \substitutefont{LGR}{\sfdefault}{cmss}
  \substitutefont{LGR}{\ttdefault}{cmtt}
\fi
\expandafter\ifx\csname T@X2\endcsname\relax
  \expandafter\ifx\csname T@T2A\endcsname\relax
  \else
  % T2A was declared as font encoding
    \substitutefont{T2A}{\rmdefault}{cmr}
    \substitutefont{T2A}{\sfdefault}{cmss}
    \substitutefont{T2A}{\ttdefault}{cmtt}
  \fi
\else
% X2 was declared as font encoding
  \substitutefont{X2}{\rmdefault}{cmr}
  \substitutefont{X2}{\sfdefault}{cmss}
  \substitutefont{X2}{\ttdefault}{cmtt}
\fi


\usepackage[Bjarne]{fncychap}
\usepackage{sphinx}

\fvset{fontsize=\small}
\usepackage{geometry}

% Include hyperref last.
\usepackage{hyperref}
% Fix anchor placement for figures with captions.
\usepackage{hypcap}% it must be loaded after hyperref.
% Set up styles of URL: it should be placed after hyperref.
\urlstyle{same}
\addto\captionsenglish{\renewcommand{\contentsname}{Contents:}}

\usepackage{sphinxmessages}
\setcounter{tocdepth}{3}
\setcounter{secnumdepth}{3}


\title{Tvilling Digital}
\date{Jun 11, 2019}
\release{}
\author{Simen Norderud Jensen}
\newcommand{\sphinxlogo}{\vbox{}}
\renewcommand{\releasename}{}
\makeindex
\begin{document}

\pagestyle{empty}
\sphinxmaketitle
\pagestyle{plain}
\sphinxtableofcontents
\pagestyle{normal}
\phantomsection\label{\detokenize{index::doc}}



\chapter{main module}
\label{\detokenize{main:module-main}}\label{\detokenize{main:main-module}}\label{\detokenize{main::doc}}\index{main (module)@\spxentry{main}\spxextra{module}}
The start point of the application.
\index{Settings (class in main)@\spxentry{Settings}\spxextra{class in main}}

\begin{fulllineitems}
\phantomsection\label{\detokenize{main:main.Settings}}\pysiglinewithargsret{\sphinxbfcode{\sphinxupquote{class }}\sphinxcode{\sphinxupquote{main.}}\sphinxbfcode{\sphinxupquote{Settings}}}{\emph{settings\_module}}{}
Bases: \sphinxcode{\sphinxupquote{object}}

A class for holding the application settings

\end{fulllineitems}

\index{main() (in module main)@\spxentry{main()}\spxextra{in module main}}

\begin{fulllineitems}
\phantomsection\label{\detokenize{main:main.main}}\pysiglinewithargsret{\sphinxcode{\sphinxupquote{main.}}\sphinxbfcode{\sphinxupquote{main}}}{\emph{args}}{}
Start the application.

Will be called with command line args if the file is run as a script

\end{fulllineitems}



\chapter{src package}
\label{\detokenize{src:src-package}}\label{\detokenize{src::doc}}

\section{Subpackages}
\label{\detokenize{src:subpackages}}

\subsection{src.blueprints package}
\label{\detokenize{src.blueprints:src-blueprints-package}}\label{\detokenize{src.blueprints::doc}}

\subsubsection{Submodules}
\label{\detokenize{src.blueprints:submodules}}

\subsubsection{src.blueprints.views module}
\label{\detokenize{src.blueprints:module-src.blueprints.views}}\label{\detokenize{src.blueprints:src-blueprints-views-module}}\index{src.blueprints.views (module)@\spxentry{src.blueprints.views}\spxextra{module}}\index{blueprint\_detail() (in module src.blueprints.views)@\spxentry{blueprint\_detail()}\spxextra{in module src.blueprints.views}}

\begin{fulllineitems}
\phantomsection\label{\detokenize{src.blueprints:src.blueprints.views.blueprint_detail}}\pysiglinewithargsret{\sphinxbfcode{\sphinxupquote{async }}\sphinxcode{\sphinxupquote{src.blueprints.views.}}\sphinxbfcode{\sphinxupquote{blueprint\_detail}}}{\emph{request: aiohttp.web\_request.Request}}{}
Get detailed information for the blueprint with the given id

\end{fulllineitems}

\index{blueprint\_list() (in module src.blueprints.views)@\spxentry{blueprint\_list()}\spxextra{in module src.blueprints.views}}

\begin{fulllineitems}
\phantomsection\label{\detokenize{src.blueprints:src.blueprints.views.blueprint_list}}\pysiglinewithargsret{\sphinxbfcode{\sphinxupquote{async }}\sphinxcode{\sphinxupquote{src.blueprints.views.}}\sphinxbfcode{\sphinxupquote{blueprint\_list}}}{\emph{request: aiohttp.web\_request.Request}}{}
List all uploaded blueprints.

Append a blueprint id to get more information about a listed blueprint.

\end{fulllineitems}

\index{dumps() (in module src.blueprints.views)@\spxentry{dumps()}\spxextra{in module src.blueprints.views}}

\begin{fulllineitems}
\phantomsection\label{\detokenize{src.blueprints:src.blueprints.views.dumps}}\pysiglinewithargsret{\sphinxcode{\sphinxupquote{src.blueprints.views.}}\sphinxbfcode{\sphinxupquote{dumps}}}{\emph{obj}, \emph{*}, \emph{skipkeys=False}, \emph{ensure\_ascii=True}, \emph{check\_circular=True}, \emph{allow\_nan=True}, \emph{cls=None}, \emph{indent=None}, \emph{separators=None}, \emph{default=\textless{}function make\_serializable\textgreater{}}, \emph{sort\_keys=False}, \emph{**kw}}{}
A version of json.dumps that uses make serializable recursively to make objects serializable

\end{fulllineitems}

\index{retrieve\_method\_info() (in module src.blueprints.views)@\spxentry{retrieve\_method\_info()}\spxextra{in module src.blueprints.views}}

\begin{fulllineitems}
\phantomsection\label{\detokenize{src.blueprints:src.blueprints.views.retrieve_method_info}}\pysiglinewithargsret{\sphinxbfcode{\sphinxupquote{async }}\sphinxcode{\sphinxupquote{src.blueprints.views.}}\sphinxbfcode{\sphinxupquote{retrieve\_method\_info}}}{\emph{class\_body}, \emph{method\_name}, \emph{params\_ignore=1}}{{ $\rightarrow$ Tuple{[}str, List{]}}}
Retrieves docs and parameters from the method
\begin{quote}\begin{description}
\item[{Parameters}] \leavevmode\begin{itemize}
\item {} 
\sphinxstyleliteralstrong{\sphinxupquote{class\_body}} \textendash{} the body of the class the method belongs to

\item {} 
\sphinxstyleliteralstrong{\sphinxupquote{method\_name}} \textendash{} the name of the method

\item {} 
\sphinxstyleliteralstrong{\sphinxupquote{params\_ignore}} \textendash{} how many of the first params to ignore, defaults to 1 (only ignore self)

\end{itemize}

\item[{Returns}] \leavevmode
a tuple containing both the docstring of the method and a list of parameters with name and default value

\end{description}\end{quote}

\end{fulllineitems}



\subsection{src.clients package}
\label{\detokenize{src.clients:src-clients-package}}\label{\detokenize{src.clients::doc}}

\subsubsection{Submodules}
\label{\detokenize{src.clients:submodules}}

\subsubsection{src.clients.models module}
\label{\detokenize{src.clients:module-src.clients.models}}\label{\detokenize{src.clients:src-clients-models-module}}\index{src.clients.models (module)@\spxentry{src.clients.models}\spxextra{module}}\index{Client (class in src.clients.models)@\spxentry{Client}\spxextra{class in src.clients.models}}

\begin{fulllineitems}
\phantomsection\label{\detokenize{src.clients:src.clients.models.Client}}\pysigline{\sphinxbfcode{\sphinxupquote{class }}\sphinxcode{\sphinxupquote{src.clients.models.}}\sphinxbfcode{\sphinxupquote{Client}}}
Bases: \sphinxcode{\sphinxupquote{object}}

Handles connections to a clients websocket connections
\index{close() (src.clients.models.Client method)@\spxentry{close()}\spxextra{src.clients.models.Client method}}

\begin{fulllineitems}
\phantomsection\label{\detokenize{src.clients:src.clients.models.Client.close}}\pysiglinewithargsret{\sphinxbfcode{\sphinxupquote{async }}\sphinxbfcode{\sphinxupquote{close}}}{}{}
Will close all the clients websocket connections

\end{fulllineitems}

\index{dict\_repr() (src.clients.models.Client method)@\spxentry{dict\_repr()}\spxextra{src.clients.models.Client method}}

\begin{fulllineitems}
\phantomsection\label{\detokenize{src.clients:src.clients.models.Client.dict_repr}}\pysiglinewithargsret{\sphinxbfcode{\sphinxupquote{dict\_repr}}}{}{{ $\rightarrow$ dict}}
Returns a the number of connections the client has

\end{fulllineitems}

\index{receive() (src.clients.models.Client method)@\spxentry{receive()}\spxextra{src.clients.models.Client method}}

\begin{fulllineitems}
\phantomsection\label{\detokenize{src.clients:src.clients.models.Client.receive}}\pysiglinewithargsret{\sphinxbfcode{\sphinxupquote{async }}\sphinxbfcode{\sphinxupquote{receive}}}{\emph{topic}, \emph{bytes}}{}
Asynchronously transmit data to the clients websocket connections

Will add the data to the buffer and send it when the buffer becomes large enough
\begin{quote}\begin{description}
\item[{Parameters}] \leavevmode\begin{itemize}
\item {} 
\sphinxstyleliteralstrong{\sphinxupquote{topic}} \textendash{} the topic the data received from

\item {} 
\sphinxstyleliteralstrong{\sphinxupquote{bytes}} \textendash{} the data received as bytes

\end{itemize}

\end{description}\end{quote}

\end{fulllineitems}


\end{fulllineitems}



\subsubsection{src.clients.views module}
\label{\detokenize{src.clients:module-src.clients.views}}\label{\detokenize{src.clients:src-clients-views-module}}\index{src.clients.views (module)@\spxentry{src.clients.views}\spxextra{module}}\index{client() (in module src.clients.views)@\spxentry{client()}\spxextra{in module src.clients.views}}

\begin{fulllineitems}
\phantomsection\label{\detokenize{src.clients:src.clients.views.client}}\pysiglinewithargsret{\sphinxbfcode{\sphinxupquote{async }}\sphinxcode{\sphinxupquote{src.clients.views.}}\sphinxbfcode{\sphinxupquote{client}}}{\emph{request: aiohttp.web\_request.Request}}{}
Show info about the client sending the request

\end{fulllineitems}

\index{dumps() (in module src.clients.views)@\spxentry{dumps()}\spxextra{in module src.clients.views}}

\begin{fulllineitems}
\phantomsection\label{\detokenize{src.clients:src.clients.views.dumps}}\pysiglinewithargsret{\sphinxcode{\sphinxupquote{src.clients.views.}}\sphinxbfcode{\sphinxupquote{dumps}}}{\emph{obj}, \emph{*}, \emph{skipkeys=False}, \emph{ensure\_ascii=True}, \emph{check\_circular=True}, \emph{allow\_nan=True}, \emph{cls=None}, \emph{indent=None}, \emph{separators=None}, \emph{default=\textless{}function make\_serializable\textgreater{}}, \emph{sort\_keys=False}, \emph{**kw}}{}
A version of json.dumps that uses make serializable recursively to make objects serializable

\end{fulllineitems}



\subsection{src.datasources package}
\label{\detokenize{src.datasources:src-datasources-package}}\label{\detokenize{src.datasources::doc}}

\subsubsection{Submodules}
\label{\detokenize{src.datasources:submodules}}

\subsubsection{src.datasources.models module}
\label{\detokenize{src.datasources:module-src.datasources.models}}\label{\detokenize{src.datasources:src-datasources-models-module}}\index{src.datasources.models (module)@\spxentry{src.datasources.models}\spxextra{module}}\index{UdpDatasource (class in src.datasources.models)@\spxentry{UdpDatasource}\spxextra{class in src.datasources.models}}

\begin{fulllineitems}
\phantomsection\label{\detokenize{src.datasources:src.datasources.models.UdpDatasource}}\pysiglinewithargsret{\sphinxbfcode{\sphinxupquote{class }}\sphinxcode{\sphinxupquote{src.datasources.models.}}\sphinxbfcode{\sphinxupquote{UdpDatasource}}}{\emph{addr: Tuple{[}str, int{]}, input\_byte\_format: str, input\_names: List{[}str{]}, output\_refs: List{[}int{]}, time\_index: int, topic: str = None}}{}
Bases: \sphinxcode{\sphinxupquote{object}}

Represents a single UDP datasource

\end{fulllineitems}

\index{UdpReceiver (class in src.datasources.models)@\spxentry{UdpReceiver}\spxextra{class in src.datasources.models}}

\begin{fulllineitems}
\phantomsection\label{\detokenize{src.datasources:src.datasources.models.UdpReceiver}}\pysiglinewithargsret{\sphinxbfcode{\sphinxupquote{class }}\sphinxcode{\sphinxupquote{src.datasources.models.}}\sphinxbfcode{\sphinxupquote{UdpReceiver}}}{\emph{kafka\_addr: str}}{}
Bases: \sphinxcode{\sphinxupquote{asyncio.protocols.DatagramProtocol}}

Handles all UDP datasources
\index{connection\_lost() (src.datasources.models.UdpReceiver method)@\spxentry{connection\_lost()}\spxextra{src.datasources.models.UdpReceiver method}}

\begin{fulllineitems}
\phantomsection\label{\detokenize{src.datasources:src.datasources.models.UdpReceiver.connection_lost}}\pysiglinewithargsret{\sphinxbfcode{\sphinxupquote{connection\_lost}}}{\emph{exc: Optional{[}Exception{]}}}{{ $\rightarrow$ None}}
Called when the connection is lost or closed.

The argument is an exception object or None (the latter
meaning a regular EOF is received or the connection was
aborted or closed).

\end{fulllineitems}

\index{connection\_made() (src.datasources.models.UdpReceiver method)@\spxentry{connection\_made()}\spxextra{src.datasources.models.UdpReceiver method}}

\begin{fulllineitems}
\phantomsection\label{\detokenize{src.datasources:src.datasources.models.UdpReceiver.connection_made}}\pysiglinewithargsret{\sphinxbfcode{\sphinxupquote{connection\_made}}}{\emph{transport: asyncio.transports.BaseTransport}}{{ $\rightarrow$ None}}
Called when a connection is made.

The argument is the transport representing the pipe connection.
To receive data, wait for data\_received() calls.
When the connection is closed, connection\_lost() is called.

\end{fulllineitems}

\index{datagram\_received() (src.datasources.models.UdpReceiver method)@\spxentry{datagram\_received()}\spxextra{src.datasources.models.UdpReceiver method}}

\begin{fulllineitems}
\phantomsection\label{\detokenize{src.datasources:src.datasources.models.UdpReceiver.datagram_received}}\pysiglinewithargsret{\sphinxbfcode{\sphinxupquote{datagram\_received}}}{\emph{raw\_data: bytes, addr: Tuple{[}str, int{]}}}{{ $\rightarrow$ None}}
Filters, transforms and buffers incoming packets before sending it to kafka

\end{fulllineitems}

\index{error\_received() (src.datasources.models.UdpReceiver method)@\spxentry{error\_received()}\spxextra{src.datasources.models.UdpReceiver method}}

\begin{fulllineitems}
\phantomsection\label{\detokenize{src.datasources:src.datasources.models.UdpReceiver.error_received}}\pysiglinewithargsret{\sphinxbfcode{\sphinxupquote{error\_received}}}{\emph{exc: Exception}}{{ $\rightarrow$ None}}
Called when a send or receive operation raises an OSError.

(Other than BlockingIOError or InterruptedError.)

\end{fulllineitems}

\index{get\_sources() (src.datasources.models.UdpReceiver method)@\spxentry{get\_sources()}\spxextra{src.datasources.models.UdpReceiver method}}

\begin{fulllineitems}
\phantomsection\label{\detokenize{src.datasources:src.datasources.models.UdpReceiver.get_sources}}\pysiglinewithargsret{\sphinxbfcode{\sphinxupquote{get\_sources}}}{}{}
Returns a list of the current sources

\end{fulllineitems}

\index{set\_source() (src.datasources.models.UdpReceiver method)@\spxentry{set\_source()}\spxextra{src.datasources.models.UdpReceiver method}}

\begin{fulllineitems}
\phantomsection\label{\detokenize{src.datasources:src.datasources.models.UdpReceiver.set_source}}\pysiglinewithargsret{\sphinxbfcode{\sphinxupquote{set\_source}}}{\emph{source\_id: str, addr: Tuple{[}str, int{]}, topic: str, input\_byte\_format: str, input\_names: List{[}str{]}, output\_refs: List{[}int{]}, time\_index: int}}{{ $\rightarrow$ None}}
Creates a new datasource object and adds it to sources, overwriting if necessary
\begin{quote}\begin{description}
\item[{Parameters}] \leavevmode\begin{itemize}
\item {} 
\sphinxstyleliteralstrong{\sphinxupquote{source\_id}} \textendash{} the id to use for the datasource

\item {} 
\sphinxstyleliteralstrong{\sphinxupquote{addr}} \textendash{} the address the datasource will send from

\item {} 
\sphinxstyleliteralstrong{\sphinxupquote{topic}} \textendash{} the topic the data will be put on

\item {} 
\sphinxstyleliteralstrong{\sphinxupquote{input\_byte\_format}} \textendash{} the byte\_format of the data that will be received

\item {} 
\sphinxstyleliteralstrong{\sphinxupquote{input\_names}} \textendash{} the names of the values in the data that will be received

\item {} 
\sphinxstyleliteralstrong{\sphinxupquote{output\_refs}} \textendash{} the indices of the values that will be transmitted to the topic

\item {} 
\sphinxstyleliteralstrong{\sphinxupquote{time\_index}} \textendash{} the index of the value that represents the time of the data

\end{itemize}

\end{description}\end{quote}

\end{fulllineitems}


\end{fulllineitems}

\index{generate\_catman\_outputs() (in module src.datasources.models)@\spxentry{generate\_catman\_outputs()}\spxextra{in module src.datasources.models}}

\begin{fulllineitems}
\phantomsection\label{\detokenize{src.datasources:src.datasources.models.generate_catman_outputs}}\pysiglinewithargsret{\sphinxcode{\sphinxupquote{src.datasources.models.}}\sphinxbfcode{\sphinxupquote{generate\_catman\_outputs}}}{\emph{output\_names: List{[}str{]}, output\_refs, single: bool = False}}{{ $\rightarrow$ Tuple{[}List{[}str{]}, List{[}int{]}, str{]}}}
Generate ouput setup for a datasource that is using the Catman software
\begin{quote}\begin{description}
\item[{Parameters}] \leavevmode\begin{itemize}
\item {} 
\sphinxstyleliteralstrong{\sphinxupquote{single}} \textendash{} true if the data from Catman is single precision (4 bytes each)

\item {} 
\sphinxstyleliteralstrong{\sphinxupquote{output\_names}} \textendash{} a list of the names of the input data

\end{itemize}

\end{description}\end{quote}

\end{fulllineitems}



\subsubsection{src.datasources.views module}
\label{\detokenize{src.datasources:module-src.datasources.views}}\label{\detokenize{src.datasources:src-datasources-views-module}}\index{src.datasources.views (module)@\spxentry{src.datasources.views}\spxextra{module}}\index{datasource\_create() (in module src.datasources.views)@\spxentry{datasource\_create()}\spxextra{in module src.datasources.views}}

\begin{fulllineitems}
\phantomsection\label{\detokenize{src.datasources:src.datasources.views.datasource_create}}\pysiglinewithargsret{\sphinxbfcode{\sphinxupquote{async }}\sphinxcode{\sphinxupquote{src.datasources.views.}}\sphinxbfcode{\sphinxupquote{datasource\_create}}}{\emph{request: aiohttp.web\_request.Request}}{}
Create a new datasource from post request.

Post parameters:
\begin{itemize}
\item {} 
id: the id to use for the source

\item {} 
address: the address to receive data from

\item {} 
port: the port to receive data from

\item {} 
output\_name: the names of the outputs
Must be all the outputs and in the same order as in the byte stream.

\item {} 
output\_ref: the indexes of the outputs that will be used

\item {} 
time\_index: the index of the time value in the output\_name list

\item {} 
byte\_format: the python struct format string for the data received.
Must include byte order (\sphinxurl{https://docs.python.org/3/library/struct.html?highlight=struct\#byte-order-size-and-alignment})
Must be in the same order as name.
Will not be used if catman is true.

\item {} 
catman: set to true to use catman byte format
byte\_format is not required if set

\item {} 
single: set to true if the data is single precision float
Only used if catman is set to true

\end{itemize}

returns redirect to created simulation page

\end{fulllineitems}

\index{datasource\_delete() (in module src.datasources.views)@\spxentry{datasource\_delete()}\spxextra{in module src.datasources.views}}

\begin{fulllineitems}
\phantomsection\label{\detokenize{src.datasources:src.datasources.views.datasource_delete}}\pysiglinewithargsret{\sphinxbfcode{\sphinxupquote{async }}\sphinxcode{\sphinxupquote{src.datasources.views.}}\sphinxbfcode{\sphinxupquote{datasource\_delete}}}{\emph{request: aiohttp.web\_request.Request}}{}
Delete the datasource

\end{fulllineitems}

\index{datasource\_detail() (in module src.datasources.views)@\spxentry{datasource\_detail()}\spxextra{in module src.datasources.views}}

\begin{fulllineitems}
\phantomsection\label{\detokenize{src.datasources:src.datasources.views.datasource_detail}}\pysiglinewithargsret{\sphinxbfcode{\sphinxupquote{async }}\sphinxcode{\sphinxupquote{src.datasources.views.}}\sphinxbfcode{\sphinxupquote{datasource\_detail}}}{\emph{request: aiohttp.web\_request.Request}}{}
Information about the datasource with the given id.
To delete the datasource append /delete
To subscribe to the datasource append /subscribe
To start the datasource append /start
To stop the datasource append /stop

\end{fulllineitems}

\index{datasource\_list() (in module src.datasources.views)@\spxentry{datasource\_list()}\spxextra{in module src.datasources.views}}

\begin{fulllineitems}
\phantomsection\label{\detokenize{src.datasources:src.datasources.views.datasource_list}}\pysiglinewithargsret{\sphinxbfcode{\sphinxupquote{async }}\sphinxcode{\sphinxupquote{src.datasources.views.}}\sphinxbfcode{\sphinxupquote{datasource\_list}}}{\emph{request: aiohttp.web\_request.Request}}{}
List all datasources.

Listed datasources will contain true if currently running and false otherwise.
Append an id to get more information about a listed datasource.
Append /create to create a new datasource

\end{fulllineitems}

\index{datasource\_start() (in module src.datasources.views)@\spxentry{datasource\_start()}\spxextra{in module src.datasources.views}}

\begin{fulllineitems}
\phantomsection\label{\detokenize{src.datasources:src.datasources.views.datasource_start}}\pysiglinewithargsret{\sphinxbfcode{\sphinxupquote{async }}\sphinxcode{\sphinxupquote{src.datasources.views.}}\sphinxbfcode{\sphinxupquote{datasource\_start}}}{\emph{request: aiohttp.web\_request.Request}}{}
Start the datasource

\end{fulllineitems}

\index{datasource\_stop() (in module src.datasources.views)@\spxentry{datasource\_stop()}\spxextra{in module src.datasources.views}}

\begin{fulllineitems}
\phantomsection\label{\detokenize{src.datasources:src.datasources.views.datasource_stop}}\pysiglinewithargsret{\sphinxbfcode{\sphinxupquote{async }}\sphinxcode{\sphinxupquote{src.datasources.views.}}\sphinxbfcode{\sphinxupquote{datasource\_stop}}}{\emph{request: aiohttp.web\_request.Request}}{}
Stop the server from retrieving data from the datasource with the given id.

\end{fulllineitems}

\index{datasource\_subscribe() (in module src.datasources.views)@\spxentry{datasource\_subscribe()}\spxextra{in module src.datasources.views}}

\begin{fulllineitems}
\phantomsection\label{\detokenize{src.datasources:src.datasources.views.datasource_subscribe}}\pysiglinewithargsret{\sphinxbfcode{\sphinxupquote{async }}\sphinxcode{\sphinxupquote{src.datasources.views.}}\sphinxbfcode{\sphinxupquote{datasource\_subscribe}}}{\emph{request: aiohttp.web\_request.Request}}{}
Subscribe to the datasource with the given id

\end{fulllineitems}

\index{datasource\_unsubscribe() (in module src.datasources.views)@\spxentry{datasource\_unsubscribe()}\spxextra{in module src.datasources.views}}

\begin{fulllineitems}
\phantomsection\label{\detokenize{src.datasources:src.datasources.views.datasource_unsubscribe}}\pysiglinewithargsret{\sphinxbfcode{\sphinxupquote{async }}\sphinxcode{\sphinxupquote{src.datasources.views.}}\sphinxbfcode{\sphinxupquote{datasource\_unsubscribe}}}{\emph{request: aiohttp.web\_request.Request}}{}
Unsubscribe to the datasource with the given id

\end{fulllineitems}

\index{dumps() (in module src.datasources.views)@\spxentry{dumps()}\spxextra{in module src.datasources.views}}

\begin{fulllineitems}
\phantomsection\label{\detokenize{src.datasources:src.datasources.views.dumps}}\pysiglinewithargsret{\sphinxcode{\sphinxupquote{src.datasources.views.}}\sphinxbfcode{\sphinxupquote{dumps}}}{\emph{obj}, \emph{*}, \emph{skipkeys=False}, \emph{ensure\_ascii=True}, \emph{check\_circular=True}, \emph{allow\_nan=True}, \emph{cls=None}, \emph{indent=None}, \emph{separators=None}, \emph{default=\textless{}function make\_serializable\textgreater{}}, \emph{sort\_keys=False}, \emph{**kw}}{}
A version of json.dumps that uses make serializable recursively to make objects serializable

\end{fulllineitems}

\index{try\_get\_source() (in module src.datasources.views)@\spxentry{try\_get\_source()}\spxextra{in module src.datasources.views}}

\begin{fulllineitems}
\phantomsection\label{\detokenize{src.datasources:src.datasources.views.try_get_source}}\pysiglinewithargsret{\sphinxcode{\sphinxupquote{src.datasources.views.}}\sphinxbfcode{\sphinxupquote{try\_get\_source}}}{\emph{app}, \emph{topic}}{}
Attempt to get the datasource sending to the given topic

Raises an HTTPNotFound error if not found.

\end{fulllineitems}



\subsection{src.fmus package}
\label{\detokenize{src.fmus:src-fmus-package}}\label{\detokenize{src.fmus::doc}}

\subsubsection{Submodules}
\label{\detokenize{src.fmus:submodules}}

\subsubsection{src.fmus.views module}
\label{\detokenize{src.fmus:module-src.fmus.views}}\label{\detokenize{src.fmus:src-fmus-views-module}}\index{src.fmus.views (module)@\spxentry{src.fmus.views}\spxextra{module}}\index{dumps() (in module src.fmus.views)@\spxentry{dumps()}\spxextra{in module src.fmus.views}}

\begin{fulllineitems}
\phantomsection\label{\detokenize{src.fmus:src.fmus.views.dumps}}\pysiglinewithargsret{\sphinxcode{\sphinxupquote{src.fmus.views.}}\sphinxbfcode{\sphinxupquote{dumps}}}{\emph{obj}, \emph{*}, \emph{skipkeys=False}, \emph{ensure\_ascii=True}, \emph{check\_circular=True}, \emph{allow\_nan=True}, \emph{cls=None}, \emph{indent=None}, \emph{separators=None}, \emph{default=\textless{}function make\_serializable\textgreater{}}, \emph{sort\_keys=False}, \emph{**kw}}{}
A version of json.dumps that uses make serializable recursively to make objects serializable

\end{fulllineitems}

\index{fmu\_detail() (in module src.fmus.views)@\spxentry{fmu\_detail()}\spxextra{in module src.fmus.views}}

\begin{fulllineitems}
\phantomsection\label{\detokenize{src.fmus:src.fmus.views.fmu_detail}}\pysiglinewithargsret{\sphinxbfcode{\sphinxupquote{async }}\sphinxcode{\sphinxupquote{src.fmus.views.}}\sphinxbfcode{\sphinxupquote{fmu\_detail}}}{\emph{request: aiohttp.web\_request.Request}}{}
Get detailed information for the FMU with the given id

Append /models to get the 3d models if any

\end{fulllineitems}

\index{fmu\_list() (in module src.fmus.views)@\spxentry{fmu\_list()}\spxextra{in module src.fmus.views}}

\begin{fulllineitems}
\phantomsection\label{\detokenize{src.fmus:src.fmus.views.fmu_list}}\pysiglinewithargsret{\sphinxbfcode{\sphinxupquote{async }}\sphinxcode{\sphinxupquote{src.fmus.views.}}\sphinxbfcode{\sphinxupquote{fmu\_list}}}{\emph{request: aiohttp.web\_request.Request}}{}
List all uploaded FMUs.

Append an FMU id to get more information about a listed FMU.

\end{fulllineitems}

\index{fmu\_model() (in module src.fmus.views)@\spxentry{fmu\_model()}\spxextra{in module src.fmus.views}}

\begin{fulllineitems}
\phantomsection\label{\detokenize{src.fmus:src.fmus.views.fmu_model}}\pysiglinewithargsret{\sphinxbfcode{\sphinxupquote{async }}\sphinxcode{\sphinxupquote{src.fmus.views.}}\sphinxbfcode{\sphinxupquote{fmu\_model}}}{\emph{request: aiohttp.web\_request.Request}}{}
Get a 3d model belonging to the FMU if it exists

\end{fulllineitems}

\index{fmu\_models() (in module src.fmus.views)@\spxentry{fmu\_models()}\spxextra{in module src.fmus.views}}

\begin{fulllineitems}
\phantomsection\label{\detokenize{src.fmus:src.fmus.views.fmu_models}}\pysiglinewithargsret{\sphinxbfcode{\sphinxupquote{async }}\sphinxcode{\sphinxupquote{src.fmus.views.}}\sphinxbfcode{\sphinxupquote{fmu\_models}}}{\emph{request: aiohttp.web\_request.Request}}{}
List the 3d models belonging to the FMU if any exists

Append the models id the get a specific model

\end{fulllineitems}



\subsection{src.processors package}
\label{\detokenize{src.processors:src-processors-package}}\label{\detokenize{src.processors::doc}}

\subsubsection{Submodules}
\label{\detokenize{src.processors:submodules}}

\subsubsection{src.processors.models module}
\label{\detokenize{src.processors:module-src.processors.models}}\label{\detokenize{src.processors:src-processors-models-module}}\index{src.processors.models (module)@\spxentry{src.processors.models}\spxextra{module}}\index{Processor (class in src.processors.models)@\spxentry{Processor}\spxextra{class in src.processors.models}}

\begin{fulllineitems}
\phantomsection\label{\detokenize{src.processors:src.processors.models.Processor}}\pysiglinewithargsret{\sphinxbfcode{\sphinxupquote{class }}\sphinxcode{\sphinxupquote{src.processors.models.}}\sphinxbfcode{\sphinxupquote{Processor}}}{\emph{processor\_id: str}, \emph{blueprint\_id: str}, \emph{blueprint\_path: str}, \emph{init\_params: dict}, \emph{topic: str}, \emph{source\_topic: str}, \emph{source\_format: str}, \emph{min\_input\_spacing: float}, \emph{min\_step\_spacing: float}, \emph{min\_output\_spacing: float}, \emph{processor\_root\_dir: str}, \emph{kafka\_server: str}}{}
Bases: \sphinxcode{\sphinxupquote{object}}

The main process endpoint for processor processes
\index{retrieve\_status() (src.processors.models.Processor method)@\spxentry{retrieve\_status()}\spxextra{src.processors.models.Processor method}}

\begin{fulllineitems}
\phantomsection\label{\detokenize{src.processors:src.processors.models.Processor.retrieve_status}}\pysiglinewithargsret{\sphinxbfcode{\sphinxupquote{retrieve\_status}}}{}{}
Retrieves the status of the processor process

Can only be called after initialization.
Should be run in a separate thread to prevent the connection from blocking the main thread
:return: the processors status as a dict

\end{fulllineitems}

\index{set\_inputs() (src.processors.models.Processor method)@\spxentry{set\_inputs()}\spxextra{src.processors.models.Processor method}}

\begin{fulllineitems}
\phantomsection\label{\detokenize{src.processors:src.processors.models.Processor.set_inputs}}\pysiglinewithargsret{\sphinxbfcode{\sphinxupquote{set\_inputs}}}{\emph{input\_refs}, \emph{measurement\_refs}, \emph{measurement\_proportions}}{}
Sets the input values, must not be called before start
\begin{quote}\begin{description}
\item[{Parameters}] \leavevmode
\sphinxstyleliteralstrong{\sphinxupquote{output\_refs}} \textendash{} the indices of the inputs that will be used

\end{description}\end{quote}

\end{fulllineitems}

\index{set\_outputs() (src.processors.models.Processor method)@\spxentry{set\_outputs()}\spxextra{src.processors.models.Processor method}}

\begin{fulllineitems}
\phantomsection\label{\detokenize{src.processors:src.processors.models.Processor.set_outputs}}\pysiglinewithargsret{\sphinxbfcode{\sphinxupquote{set\_outputs}}}{\emph{output\_refs}}{}
Sets the output values, must not be called before start
\begin{quote}\begin{description}
\item[{Parameters}] \leavevmode\begin{itemize}
\item {} 
\sphinxstyleliteralstrong{\sphinxupquote{input\_refs}} \textendash{} the indices of the inputs that will be used

\item {} 
\sphinxstyleliteralstrong{\sphinxupquote{measurement\_refs}} \textendash{} the indices of the input data values that will be used.
Must be in the same order as input\_ref.

\item {} 
\sphinxstyleliteralstrong{\sphinxupquote{measurement\_proportions}} \textendash{} list of scales to be used on values before inputting them.
Must be in the same order as input\_ref.

\end{itemize}

\end{description}\end{quote}

\end{fulllineitems}

\index{start() (src.processors.models.Processor method)@\spxentry{start()}\spxextra{src.processors.models.Processor method}}

\begin{fulllineitems}
\phantomsection\label{\detokenize{src.processors:src.processors.models.Processor.start}}\pysiglinewithargsret{\sphinxbfcode{\sphinxupquote{start}}}{\emph{input\_refs}, \emph{measurement\_refs}, \emph{measurement\_proportions}, \emph{output\_refs}, \emph{start\_params}}{}
Starts the process, must not be called before init\_results
\begin{quote}\begin{description}
\item[{Parameters}] \leavevmode\begin{itemize}
\item {} 
\sphinxstyleliteralstrong{\sphinxupquote{input\_refs}} \textendash{} the indices of the inputs that will be used

\item {} 
\sphinxstyleliteralstrong{\sphinxupquote{measurement\_refs}} \textendash{} the indices of the input data values that will be used.
Must be in the same order as input\_ref.

\item {} 
\sphinxstyleliteralstrong{\sphinxupquote{measurement\_proportions}} \textendash{} list of scales to be used on values before inputting them.
Must be in the same order as input\_ref.

\item {} 
\sphinxstyleliteralstrong{\sphinxupquote{output\_refs}} \textendash{} the indices of the inputs that will be used

\item {} 
\sphinxstyleliteralstrong{\sphinxupquote{start\_params}} \textendash{} the processors start parameters as a dict

\end{itemize}

\item[{Returns}] \leavevmode
the processors status as a dict

\end{description}\end{quote}

\end{fulllineitems}

\index{stop() (src.processors.models.Processor method)@\spxentry{stop()}\spxextra{src.processors.models.Processor method}}

\begin{fulllineitems}
\phantomsection\label{\detokenize{src.processors:src.processors.models.Processor.stop}}\pysiglinewithargsret{\sphinxbfcode{\sphinxupquote{async }}\sphinxbfcode{\sphinxupquote{stop}}}{}{}
Attempts to stop the process nicely, killing it otherwise

\end{fulllineitems}


\end{fulllineitems}

\index{Variable (class in src.processors.models)@\spxentry{Variable}\spxextra{class in src.processors.models}}

\begin{fulllineitems}
\phantomsection\label{\detokenize{src.processors:src.processors.models.Variable}}\pysiglinewithargsret{\sphinxbfcode{\sphinxupquote{class }}\sphinxcode{\sphinxupquote{src.processors.models.}}\sphinxbfcode{\sphinxupquote{Variable}}}{\emph{valueReference: int}, \emph{name: str}}{}
Bases: \sphinxcode{\sphinxupquote{object}}

A simple container class for variable attributes

\end{fulllineitems}

\index{processor\_process() (in module src.processors.models)@\spxentry{processor\_process()}\spxextra{in module src.processors.models}}

\begin{fulllineitems}
\phantomsection\label{\detokenize{src.processors:src.processors.models.processor_process}}\pysiglinewithargsret{\sphinxcode{\sphinxupquote{src.processors.models.}}\sphinxbfcode{\sphinxupquote{processor\_process}}}{\emph{connection: multiprocessing.connection.Connection}, \emph{blueprint\_path: str}, \emph{init\_params: dict}, \emph{processor\_dir: str}, \emph{topic: str}, \emph{source\_topic: str}, \emph{source\_format: str}, \emph{kafka\_server: str}, \emph{min\_input\_spacing: float}, \emph{min\_step\_spacing: float}, \emph{min\_output\_spacing: float}}{}
Runs the given blueprint as a processor

Is meant to be run in a separate process
\begin{quote}\begin{description}
\item[{Parameters}] \leavevmode\begin{itemize}
\item {} 
\sphinxstyleliteralstrong{\sphinxupquote{connection}} \textendash{} a connection object to communicate with the main process

\item {} 
\sphinxstyleliteralstrong{\sphinxupquote{blueprint\_path}} \textendash{} the path to the blueprint folder

\item {} 
\sphinxstyleliteralstrong{\sphinxupquote{init\_params}} \textendash{} the initialization parameters to the processor as a dictionary

\item {} 
\sphinxstyleliteralstrong{\sphinxupquote{processor\_dir}} \textendash{} the directory the created process will run in

\item {} 
\sphinxstyleliteralstrong{\sphinxupquote{topic}} \textendash{} the topic the process will send results to

\item {} 
\sphinxstyleliteralstrong{\sphinxupquote{source\_topic}} \textendash{} the topic the process will receive data from

\item {} 
\sphinxstyleliteralstrong{\sphinxupquote{source\_format}} \textendash{} the byte format of the data the process will receive

\item {} 
\sphinxstyleliteralstrong{\sphinxupquote{kafka\_server}} \textendash{} the address of the kafka bootstrap server the process will use

\item {} 
\sphinxstyleliteralstrong{\sphinxupquote{min\_input\_spacing}} \textendash{} the minimum time between each input to the processor

\item {} 
\sphinxstyleliteralstrong{\sphinxupquote{min\_step\_spacing}} \textendash{} the minimum time between each step function call on the processor

\item {} 
\sphinxstyleliteralstrong{\sphinxupquote{min\_output\_spacing}} \textendash{} the minimum time between each results retrieval from the processor

\end{itemize}

\item[{Returns}] \leavevmode


\end{description}\end{quote}

\end{fulllineitems}



\subsubsection{src.processors.views module}
\label{\detokenize{src.processors:module-src.processors.views}}\label{\detokenize{src.processors:src-processors-views-module}}\index{src.processors.views (module)@\spxentry{src.processors.views}\spxextra{module}}\index{dumps() (in module src.processors.views)@\spxentry{dumps()}\spxextra{in module src.processors.views}}

\begin{fulllineitems}
\phantomsection\label{\detokenize{src.processors:src.processors.views.dumps}}\pysiglinewithargsret{\sphinxcode{\sphinxupquote{src.processors.views.}}\sphinxbfcode{\sphinxupquote{dumps}}}{\emph{obj}, \emph{*}, \emph{skipkeys=False}, \emph{ensure\_ascii=True}, \emph{check\_circular=True}, \emph{allow\_nan=True}, \emph{cls=None}, \emph{indent=None}, \emph{separators=None}, \emph{default=\textless{}function make\_serializable\textgreater{}}, \emph{sort\_keys=False}, \emph{**kw}}{}
A version of json.dumps that uses make serializable recursively to make objects serializable

\end{fulllineitems}

\index{processor\_create() (in module src.processors.views)@\spxentry{processor\_create()}\spxextra{in module src.processors.views}}

\begin{fulllineitems}
\phantomsection\label{\detokenize{src.processors:src.processors.views.processor_create}}\pysiglinewithargsret{\sphinxbfcode{\sphinxupquote{async }}\sphinxcode{\sphinxupquote{src.processors.views.}}\sphinxbfcode{\sphinxupquote{processor\_create}}}{\emph{request: aiohttp.web\_request.Request}}{}
Create a new processor from post request.

Post params:
\begin{itemize}
\item {} 
id:* id of new processor instance
max 20 chars, first char must be alphabetic or underscore, other chars must be alphabetic, digit or underscore

\item {} 
blueprint:* id of blueprint to be used
max 20 chars, first char must be alphabetic or underscore, other chars must be alphabetic, digit or underscore

\item {} 
init\_params: the processor specific initialization variables as a json string

\item {} 
topic:* topic to use as input to processor

\item {} 
min\_output\_interval: the shortest time allowed between each output from processor in seconds

\end{itemize}

\end{fulllineitems}

\index{processor\_delete() (in module src.processors.views)@\spxentry{processor\_delete()}\spxextra{in module src.processors.views}}

\begin{fulllineitems}
\phantomsection\label{\detokenize{src.processors:src.processors.views.processor_delete}}\pysiglinewithargsret{\sphinxbfcode{\sphinxupquote{async }}\sphinxcode{\sphinxupquote{src.processors.views.}}\sphinxbfcode{\sphinxupquote{processor\_delete}}}{\emph{request: aiohttp.web\_request.Request}}{}
Delete the processor with the given id.

\end{fulllineitems}

\index{processor\_detail() (in module src.processors.views)@\spxentry{processor\_detail()}\spxextra{in module src.processors.views}}

\begin{fulllineitems}
\phantomsection\label{\detokenize{src.processors:src.processors.views.processor_detail}}\pysiglinewithargsret{\sphinxbfcode{\sphinxupquote{async }}\sphinxcode{\sphinxupquote{src.processors.views.}}\sphinxbfcode{\sphinxupquote{processor\_detail}}}{\emph{request: aiohttp.web\_request.Request}}{}
Get detailed information for the processor with the given id

Append /subscribe to subscribe to the processor
Append /unsubscribe to unsubscribe to the processor
Append /stop to stop the processor
Append /delete to delete the processor
Append /outputs to get the outputs of the processor
Append /inputs to get the inputs of the processor
Append /status to update and get the status of the processor

\end{fulllineitems}

\index{processor\_inputs\_update() (in module src.processors.views)@\spxentry{processor\_inputs\_update()}\spxextra{in module src.processors.views}}

\begin{fulllineitems}
\phantomsection\label{\detokenize{src.processors:src.processors.views.processor_inputs_update}}\pysiglinewithargsret{\sphinxbfcode{\sphinxupquote{async }}\sphinxcode{\sphinxupquote{src.processors.views.}}\sphinxbfcode{\sphinxupquote{processor\_inputs\_update}}}{\emph{request: aiohttp.web\_request.Request}}{}
Update the processor inputs
\begin{quote}

Post params:
\begin{itemize}
\item {} 
input\_ref: reference values to the inputs to be used

\item {} 
measurement\_ref: reference values to the measurement inputs to be used for the inputs.
Must be in the same order as input\_ref.

\item {} 
measurement\_proportion: scale to be used on measurement values before inputting them.
Must be in the same order as input\_ref.

\end{itemize}
\end{quote}

\end{fulllineitems}

\index{processor\_list() (in module src.processors.views)@\spxentry{processor\_list()}\spxextra{in module src.processors.views}}

\begin{fulllineitems}
\phantomsection\label{\detokenize{src.processors:src.processors.views.processor_list}}\pysiglinewithargsret{\sphinxbfcode{\sphinxupquote{async }}\sphinxcode{\sphinxupquote{src.processors.views.}}\sphinxbfcode{\sphinxupquote{processor\_list}}}{\emph{request: aiohttp.web\_request.Request}}{}
List all created processors.

Returns a json object of processor id to processor status objects.

Append a processor id to get more information about a listed processor.
Append /create to create a new processor instance
Append /clear to delete stopped processors

\end{fulllineitems}

\index{processor\_outputs\_update() (in module src.processors.views)@\spxentry{processor\_outputs\_update()}\spxextra{in module src.processors.views}}

\begin{fulllineitems}
\phantomsection\label{\detokenize{src.processors:src.processors.views.processor_outputs_update}}\pysiglinewithargsret{\sphinxbfcode{\sphinxupquote{async }}\sphinxcode{\sphinxupquote{src.processors.views.}}\sphinxbfcode{\sphinxupquote{processor\_outputs\_update}}}{\emph{request: aiohttp.web\_request.Request}}{}
Update the processor outputs
\begin{quote}

Post params:
\begin{itemize}
\item {} 
output\_ref: reference values to the outputs to be used

\end{itemize}
\end{quote}

\end{fulllineitems}

\index{processor\_start() (in module src.processors.views)@\spxentry{processor\_start()}\spxextra{in module src.processors.views}}

\begin{fulllineitems}
\phantomsection\label{\detokenize{src.processors:src.processors.views.processor_start}}\pysiglinewithargsret{\sphinxbfcode{\sphinxupquote{async }}\sphinxcode{\sphinxupquote{src.processors.views.}}\sphinxbfcode{\sphinxupquote{processor\_start}}}{\emph{request: aiohttp.web\_request.Request}}{}
Start a processor from post request.

Post params:
\begin{itemize}
\item {} 
id:* id of processor instance
max 20 chars, first char must be alphabetic or underscore, other chars must be alphabetic, digit or underscore

\item {} 
start\_params: the processor specific start parameters as a json string

\item {} 
input\_ref: list of reference values to the inputs to be used

\item {} 
output\_ref: list of reference values to the outputs to be used

\item {} 
measurement\_ref: list of reference values to the measurement inputs to be used for the inputs.
Must be in the same order as input\_ref.

\item {} 
measurement\_proportion: list of scales to be used on measurement values before inputting them.
Must be in the same order as input\_ref.

\end{itemize}

\end{fulllineitems}

\index{processor\_status() (in module src.processors.views)@\spxentry{processor\_status()}\spxextra{in module src.processors.views}}

\begin{fulllineitems}
\phantomsection\label{\detokenize{src.processors:src.processors.views.processor_status}}\pysiglinewithargsret{\sphinxbfcode{\sphinxupquote{async }}\sphinxcode{\sphinxupquote{src.processors.views.}}\sphinxbfcode{\sphinxupquote{processor\_status}}}{\emph{request: aiohttp.web\_request.Request}}{}
Updates and returns the current status of the processor

\end{fulllineitems}

\index{processor\_stop() (in module src.processors.views)@\spxentry{processor\_stop()}\spxextra{in module src.processors.views}}

\begin{fulllineitems}
\phantomsection\label{\detokenize{src.processors:src.processors.views.processor_stop}}\pysiglinewithargsret{\sphinxbfcode{\sphinxupquote{async }}\sphinxcode{\sphinxupquote{src.processors.views.}}\sphinxbfcode{\sphinxupquote{processor\_stop}}}{\emph{request: aiohttp.web\_request.Request}}{}
Stop the processor with the given id.

\end{fulllineitems}

\index{processor\_subscribe() (in module src.processors.views)@\spxentry{processor\_subscribe()}\spxextra{in module src.processors.views}}

\begin{fulllineitems}
\phantomsection\label{\detokenize{src.processors:src.processors.views.processor_subscribe}}\pysiglinewithargsret{\sphinxbfcode{\sphinxupquote{async }}\sphinxcode{\sphinxupquote{src.processors.views.}}\sphinxbfcode{\sphinxupquote{processor\_subscribe}}}{\emph{request: aiohttp.web\_request.Request}}{}
Subscribe to the processor with the given id

\end{fulllineitems}

\index{processor\_unsubscribe() (in module src.processors.views)@\spxentry{processor\_unsubscribe()}\spxextra{in module src.processors.views}}

\begin{fulllineitems}
\phantomsection\label{\detokenize{src.processors:src.processors.views.processor_unsubscribe}}\pysiglinewithargsret{\sphinxbfcode{\sphinxupquote{async }}\sphinxcode{\sphinxupquote{src.processors.views.}}\sphinxbfcode{\sphinxupquote{processor\_unsubscribe}}}{\emph{request: aiohttp.web\_request.Request}}{}
Unsubscribe to the processor with the given id

\end{fulllineitems}

\index{processors\_clear() (in module src.processors.views)@\spxentry{processors\_clear()}\spxextra{in module src.processors.views}}

\begin{fulllineitems}
\phantomsection\label{\detokenize{src.processors:src.processors.views.processors_clear}}\pysiglinewithargsret{\sphinxbfcode{\sphinxupquote{async }}\sphinxcode{\sphinxupquote{src.processors.views.}}\sphinxbfcode{\sphinxupquote{processors\_clear}}}{\emph{request: aiohttp.web\_request.Request}}{}
Delete data from all processors that are not running

\end{fulllineitems}

\index{retrieve\_processor\_status() (in module src.processors.views)@\spxentry{retrieve\_processor\_status()}\spxextra{in module src.processors.views}}

\begin{fulllineitems}
\phantomsection\label{\detokenize{src.processors:src.processors.views.retrieve_processor_status}}\pysiglinewithargsret{\sphinxbfcode{\sphinxupquote{async }}\sphinxcode{\sphinxupquote{src.processors.views.}}\sphinxbfcode{\sphinxupquote{retrieve\_processor\_status}}}{\emph{app}, \emph{processor\_instance}}{}
Retrieve the initialization results from a processor

Will put the results in app{[}‘topics’{]} and return them.

\end{fulllineitems}



\section{Submodules}
\label{\detokenize{src:submodules}}

\section{src.kafka module}
\label{\detokenize{src:module-src.kafka}}\label{\detokenize{src:src-kafka-module}}\index{src.kafka (module)@\spxentry{src.kafka}\spxextra{module}}\index{consume\_from\_kafka() (in module src.kafka)@\spxentry{consume\_from\_kafka()}\spxextra{in module src.kafka}}

\begin{fulllineitems}
\phantomsection\label{\detokenize{src:src.kafka.consume_from_kafka}}\pysiglinewithargsret{\sphinxbfcode{\sphinxupquote{async }}\sphinxcode{\sphinxupquote{src.kafka.}}\sphinxbfcode{\sphinxupquote{consume\_from\_kafka}}}{\emph{app: aiohttp.web\_app.Application}}{}
The function responsible for delivering data to the connected clients.

\end{fulllineitems}



\section{src.server module}
\label{\detokenize{src:module-src.server}}\label{\detokenize{src:src-server-module}}\index{src.server (module)@\spxentry{src.server}\spxextra{module}}\index{cleanup\_background\_tasks() (in module src.server)@\spxentry{cleanup\_background\_tasks()}\spxextra{in module src.server}}

\begin{fulllineitems}
\phantomsection\label{\detokenize{src:src.server.cleanup_background_tasks}}\pysiglinewithargsret{\sphinxbfcode{\sphinxupquote{async }}\sphinxcode{\sphinxupquote{src.server.}}\sphinxbfcode{\sphinxupquote{cleanup\_background\_tasks}}}{\emph{app}}{}
A method to be called on shutdown, closes the WebSocket, Kafka, and UDP connections

\end{fulllineitems}

\index{init\_app() (in module src.server)@\spxentry{init\_app()}\spxextra{in module src.server}}

\begin{fulllineitems}
\phantomsection\label{\detokenize{src:src.server.init_app}}\pysiglinewithargsret{\sphinxcode{\sphinxupquote{src.server.}}\sphinxbfcode{\sphinxupquote{init\_app}}}{\emph{settings}}{{ $\rightarrow$ aiohttp.web\_app.Application}}
Initializes and starts the server

\end{fulllineitems}

\index{start\_background\_tasks() (in module src.server)@\spxentry{start\_background\_tasks()}\spxextra{in module src.server}}

\begin{fulllineitems}
\phantomsection\label{\detokenize{src:src.server.start_background_tasks}}\pysiglinewithargsret{\sphinxbfcode{\sphinxupquote{async }}\sphinxcode{\sphinxupquote{src.server.}}\sphinxbfcode{\sphinxupquote{start\_background\_tasks}}}{\emph{app}}{}
A method to be called on startup, initiates the Kafka and UDP connections

\end{fulllineitems}



\section{src.utils module}
\label{\detokenize{src:module-src.utils}}\label{\detokenize{src:src-utils-module}}\index{src.utils (module)@\spxentry{src.utils}\spxextra{module}}\index{RouteTableDefDocs (class in src.utils)@\spxentry{RouteTableDefDocs}\spxextra{class in src.utils}}

\begin{fulllineitems}
\phantomsection\label{\detokenize{src:src.utils.RouteTableDefDocs}}\pysigline{\sphinxbfcode{\sphinxupquote{class }}\sphinxcode{\sphinxupquote{src.utils.}}\sphinxbfcode{\sphinxupquote{RouteTableDefDocs}}}
Bases: \sphinxcode{\sphinxupquote{aiohttp.web\_routedef.RouteTableDef}}

A custom RouteTableDef that also creates /docs pages with the docstring of the functions.
\index{get\_docs\_response() (src.utils.RouteTableDefDocs static method)@\spxentry{get\_docs\_response()}\spxextra{src.utils.RouteTableDefDocs static method}}

\begin{fulllineitems}
\phantomsection\label{\detokenize{src:src.utils.RouteTableDefDocs.get_docs_response}}\pysiglinewithargsret{\sphinxbfcode{\sphinxupquote{static }}\sphinxbfcode{\sphinxupquote{get\_docs\_response}}}{\emph{handler}}{}
Creates a new function that returns the docs of the given function

\end{fulllineitems}

\index{route() (src.utils.RouteTableDefDocs method)@\spxentry{route()}\spxextra{src.utils.RouteTableDefDocs method}}

\begin{fulllineitems}
\phantomsection\label{\detokenize{src:src.utils.RouteTableDefDocs.route}}\pysiglinewithargsret{\sphinxbfcode{\sphinxupquote{route}}}{\emph{method: str}, \emph{path: str}, \emph{**kwargs}}{{ $\rightarrow$ Callable{[}{[}Union{[}aiohttp.abc.AbstractView, Callable{[}{[}None{]}, Awaitable{[}None{]}{]}{]}{]}, Union{[}aiohttp.abc.AbstractView, Callable{[}{[}None{]}, Awaitable{[}None{]}{]}{]}{]}}}
Adds the given function to routes, then attempts to add the docstring of the function to /docs

\end{fulllineitems}


\end{fulllineitems}

\index{dumps() (in module src.utils)@\spxentry{dumps()}\spxextra{in module src.utils}}

\begin{fulllineitems}
\phantomsection\label{\detokenize{src:src.utils.dumps}}\pysiglinewithargsret{\sphinxcode{\sphinxupquote{src.utils.}}\sphinxbfcode{\sphinxupquote{dumps}}}{\emph{obj}, \emph{*}, \emph{skipkeys=False}, \emph{ensure\_ascii=True}, \emph{check\_circular=True}, \emph{allow\_nan=True}, \emph{cls=None}, \emph{indent=None}, \emph{separators=None}, \emph{default=\textless{}function make\_serializable\textgreater{}}, \emph{sort\_keys=False}, \emph{**kw}}{}
A version of json.dumps that uses make serializable recursively to make objects serializable

\end{fulllineitems}

\index{find\_in\_dir() (in module src.utils)@\spxentry{find\_in\_dir()}\spxextra{in module src.utils}}

\begin{fulllineitems}
\phantomsection\label{\detokenize{src:src.utils.find_in_dir}}\pysiglinewithargsret{\sphinxbfcode{\sphinxupquote{async }}\sphinxcode{\sphinxupquote{src.utils.}}\sphinxbfcode{\sphinxupquote{find\_in\_dir}}}{\emph{filename}, \emph{parent\_directory=''}}{}
Checks if the given file is present in the given directory and returns the file if found.
Raises a HTTPNotFound exception otherwise

\end{fulllineitems}

\index{get\_client() (in module src.utils)@\spxentry{get\_client()}\spxextra{in module src.utils}}

\begin{fulllineitems}
\phantomsection\label{\detokenize{src:src.utils.get_client}}\pysiglinewithargsret{\sphinxbfcode{\sphinxupquote{async }}\sphinxcode{\sphinxupquote{src.utils.}}\sphinxbfcode{\sphinxupquote{get\_client}}}{\emph{request}}{}
Returns the client object belonging to the owner of the request.

\end{fulllineitems}

\index{make\_serializable() (in module src.utils)@\spxentry{make\_serializable()}\spxextra{in module src.utils}}

\begin{fulllineitems}
\phantomsection\label{\detokenize{src:src.utils.make_serializable}}\pysiglinewithargsret{\sphinxcode{\sphinxupquote{src.utils.}}\sphinxbfcode{\sphinxupquote{make\_serializable}}}{\emph{o}}{}
Makes the given object JSON serializable by turning it into a structure of dicts and strings.

\end{fulllineitems}

\index{try\_get() (in module src.utils)@\spxentry{try\_get()}\spxextra{in module src.utils}}

\begin{fulllineitems}
\phantomsection\label{\detokenize{src:src.utils.try_get}}\pysiglinewithargsret{\sphinxcode{\sphinxupquote{src.utils.}}\sphinxbfcode{\sphinxupquote{try\_get}}}{\emph{post}, \emph{key}, \emph{parser=None}}{}
Attempt to get the value with key from post.
\begin{quote}\begin{description}
\item[{Parameters}] \leavevmode\begin{itemize}
\item {} 
\sphinxstyleliteralstrong{\sphinxupquote{post}} \textendash{} The post request the value will be retrieved from

\item {} 
\sphinxstyleliteralstrong{\sphinxupquote{key}} \textendash{} Key used to retrieve the value

\item {} 
\sphinxstyleliteralstrong{\sphinxupquote{parser}} \textendash{} Will be used to parse the retrieved value if given

\end{itemize}

\item[{Returns}] \leavevmode
The retrieved and parsed value.
Returns the first value if more than one value is found.

\item[{Raises}] \leavevmode\begin{itemize}
\item {} 
\sphinxstyleliteralstrong{\sphinxupquote{web.HTTPUnprocessableEntity}} \textendash{} If a value with the given key is not found

\item {} 
\sphinxstyleliteralstrong{\sphinxupquote{web.HTTPBadRequest}} \textendash{} If parsing of the value failed

\end{itemize}

\end{description}\end{quote}

\end{fulllineitems}

\index{try\_get\_all() (in module src.utils)@\spxentry{try\_get\_all()}\spxextra{in module src.utils}}

\begin{fulllineitems}
\phantomsection\label{\detokenize{src:src.utils.try_get_all}}\pysiglinewithargsret{\sphinxbfcode{\sphinxupquote{async }}\sphinxcode{\sphinxupquote{src.utils.}}\sphinxbfcode{\sphinxupquote{try\_get\_all}}}{\emph{post}, \emph{key}, \emph{parser=None}}{}
Attempt to get all values with the given key from the given post request.
Attempts to parse the values using the parser if a parser is given.
Raises a HTTPException if the key is not found or the parsing fails.

\end{fulllineitems}

\index{try\_get\_topic() (in module src.utils)@\spxentry{try\_get\_topic()}\spxextra{in module src.utils}}

\begin{fulllineitems}
\phantomsection\label{\detokenize{src:src.utils.try_get_topic}}\pysiglinewithargsret{\sphinxcode{\sphinxupquote{src.utils.}}\sphinxbfcode{\sphinxupquote{try\_get\_topic}}}{\emph{post}}{}
Attempt to get the topic value from the given post request.
Attempts to validate the topic value with the topic validator strings if found.
Raises a HTTPException if the key is not found or the validation fails.

\end{fulllineitems}

\index{try\_get\_validate() (in module src.utils)@\spxentry{try\_get\_validate()}\spxextra{in module src.utils}}

\begin{fulllineitems}
\phantomsection\label{\detokenize{src:src.utils.try_get_validate}}\pysiglinewithargsret{\sphinxcode{\sphinxupquote{src.utils.}}\sphinxbfcode{\sphinxupquote{try\_get\_validate}}}{\emph{post}, \emph{key}}{}
Attempt to get the value with the given key from the given post request.
Returns the first value if more than one value is found.
Attempts to validate the value with the validator strings if found.
Raises a HTTPException if the key is not found or the validation fails.

\end{fulllineitems}



\section{src.views module}
\label{\detokenize{src:module-src.views}}\label{\detokenize{src:src-views-module}}\index{src.views (module)@\spxentry{src.views}\spxextra{module}}\index{history() (in module src.views)@\spxentry{history()}\spxextra{in module src.views}}

\begin{fulllineitems}
\phantomsection\label{\detokenize{src:src.views.history}}\pysiglinewithargsret{\sphinxbfcode{\sphinxupquote{async }}\sphinxcode{\sphinxupquote{src.views.}}\sphinxbfcode{\sphinxupquote{history}}}{\emph{request: aiohttp.web\_request.Request}}{}
Get historic data from the given topic

get params:
- start: the start timestamp as milliseconds since 00:00:00 Thursday, 1 January 1970
- end: (optinoal) the end timestamp as milliseconds since 00:00:00 Thursday, 1 January 1970

\end{fulllineitems}

\index{index() (in module src.views)@\spxentry{index()}\spxextra{in module src.views}}

\begin{fulllineitems}
\phantomsection\label{\detokenize{src:src.views.index}}\pysiglinewithargsret{\sphinxbfcode{\sphinxupquote{async }}\sphinxcode{\sphinxupquote{src.views.}}\sphinxbfcode{\sphinxupquote{index}}}{\emph{request: aiohttp.web\_request.Request}}{}
The API index

A standard HTTP request will return a sample page with a simple example of api use.
A WebSocket request will initiate a websocket connection making it possible to retrieve measurement and simulation data.

Available endpoints are
- /client for information about the clients websocket connections
- /datasources/ for measurement data sources
- /processors/ for running processors on the data
- /blueprints/ for the blueprints used to create processors
- /fmus/ for available FMUs (for the fmu blueprint)
- /models/ for available models (for the fedem blueprint)
- /topics/ for all available data sources (datasources and processors)

\end{fulllineitems}

\index{models() (in module src.views)@\spxentry{models()}\spxextra{in module src.views}}

\begin{fulllineitems}
\phantomsection\label{\detokenize{src:src.views.models}}\pysiglinewithargsret{\sphinxbfcode{\sphinxupquote{async }}\sphinxcode{\sphinxupquote{src.views.}}\sphinxbfcode{\sphinxupquote{models}}}{\emph{request: aiohttp.web\_request.Request}}{}
List available models for the fedem blueprint

\end{fulllineitems}

\index{session\_endpoint() (in module src.views)@\spxentry{session\_endpoint()}\spxextra{in module src.views}}

\begin{fulllineitems}
\phantomsection\label{\detokenize{src:src.views.session_endpoint}}\pysiglinewithargsret{\sphinxbfcode{\sphinxupquote{async }}\sphinxcode{\sphinxupquote{src.views.}}\sphinxbfcode{\sphinxupquote{session\_endpoint}}}{\emph{request: aiohttp.web\_request.Request}}{}
Only returns a session cookie

Generates and returns a session cookie.

\end{fulllineitems}

\index{subscribe() (in module src.views)@\spxentry{subscribe()}\spxextra{in module src.views}}

\begin{fulllineitems}
\phantomsection\label{\detokenize{src:src.views.subscribe}}\pysiglinewithargsret{\sphinxbfcode{\sphinxupquote{async }}\sphinxcode{\sphinxupquote{src.views.}}\sphinxbfcode{\sphinxupquote{subscribe}}}{\emph{request: aiohttp.web\_request.Request}}{}
Subscribe to the given topic

\end{fulllineitems}

\index{topics() (in module src.views)@\spxentry{topics()}\spxextra{in module src.views}}

\begin{fulllineitems}
\phantomsection\label{\detokenize{src:src.views.topics}}\pysiglinewithargsret{\sphinxbfcode{\sphinxupquote{async }}\sphinxcode{\sphinxupquote{src.views.}}\sphinxbfcode{\sphinxupquote{topics}}}{\emph{request: aiohttp.web\_request.Request}}{}
Lists the available data sources for plotting or processors

Append the id of a topic to get details about only that topic
Append the id of a topic and /subscribe to subscribe to a topic
Append the id of a topic and /unsubscribe to unsubscribe to a topic
Append the id of a topic and /history to get historic data from a topic

\end{fulllineitems}

\index{topics\_detail() (in module src.views)@\spxentry{topics\_detail()}\spxextra{in module src.views}}

\begin{fulllineitems}
\phantomsection\label{\detokenize{src:src.views.topics_detail}}\pysiglinewithargsret{\sphinxbfcode{\sphinxupquote{async }}\sphinxcode{\sphinxupquote{src.views.}}\sphinxbfcode{\sphinxupquote{topics\_detail}}}{\emph{request: aiohttp.web\_request.Request}}{}
Show a single topic

Append /subscribe to subscribe to the topic
Append /unsubscribe to unsubscribe to the topic
Append /history to get historic data from a topic

\end{fulllineitems}

\index{unsubscribe() (in module src.views)@\spxentry{unsubscribe()}\spxextra{in module src.views}}

\begin{fulllineitems}
\phantomsection\label{\detokenize{src:src.views.unsubscribe}}\pysiglinewithargsret{\sphinxbfcode{\sphinxupquote{async }}\sphinxcode{\sphinxupquote{src.views.}}\sphinxbfcode{\sphinxupquote{unsubscribe}}}{\emph{request: aiohttp.web\_request.Request}}{}
Unsubscribe to the given topic

\end{fulllineitems}



\section{Module contents}
\label{\detokenize{src:module-src}}\label{\detokenize{src:module-contents}}\index{src (module)@\spxentry{src}\spxextra{module}}

\chapter{blueprints package}
\label{\detokenize{blueprints:blueprints-package}}\label{\detokenize{blueprints::doc}}

\section{Submodules}
\label{\detokenize{blueprints:submodules}}

\section{blueprints.fmu module}
\label{\detokenize{blueprints:module-files.blueprints.fmu}}\label{\detokenize{blueprints:blueprints-fmu-module}}\index{files.blueprints.fmu (module)@\spxentry{files.blueprints.fmu}\spxextra{module}}
A blueprint for running FMUs.
\index{P (class in files.blueprints.fmu)@\spxentry{P}\spxextra{class in files.blueprints.fmu}}

\begin{fulllineitems}
\phantomsection\label{\detokenize{blueprints:files.blueprints.fmu.P}}\pysiglinewithargsret{\sphinxbfcode{\sphinxupquote{class }}\sphinxcode{\sphinxupquote{files.blueprints.fmu.}}\sphinxbfcode{\sphinxupquote{P}}}{\emph{fmu='testrig.fmu'}}{}
The interface between the application and the FMU
\index{start() (files.blueprints.fmu.P method)@\spxentry{start()}\spxextra{files.blueprints.fmu.P method}}

\begin{fulllineitems}
\phantomsection\label{\detokenize{blueprints:files.blueprints.fmu.P.start}}\pysiglinewithargsret{\sphinxbfcode{\sphinxupquote{start}}}{\emph{start\_time}, \emph{time\_step\_input\_ref='-1'}}{}
Starts the FMU
\begin{quote}\begin{description}
\item[{Parameters}] \leavevmode\begin{itemize}
\item {} 
\sphinxstyleliteralstrong{\sphinxupquote{start\_time}} \textendash{} not used in this blueprint

\item {} 
\sphinxstyleliteralstrong{\sphinxupquote{time\_step\_input\_ref}} \textendash{} optional value for custom time\_step input

\end{itemize}

\end{description}\end{quote}

\end{fulllineitems}


\end{fulllineitems}

\index{prepare\_outputs() (in module files.blueprints.fmu)@\spxentry{prepare\_outputs()}\spxextra{in module files.blueprints.fmu}}

\begin{fulllineitems}
\phantomsection\label{\detokenize{blueprints:files.blueprints.fmu.prepare_outputs}}\pysiglinewithargsret{\sphinxcode{\sphinxupquote{files.blueprints.fmu.}}\sphinxbfcode{\sphinxupquote{prepare\_outputs}}}{\emph{output\_refs}}{}
Create FMUPy compatible value references and outputs buffer from output\_refs
\begin{quote}\begin{description}
\item[{Parameters}] \leavevmode
\sphinxstyleliteralstrong{\sphinxupquote{output\_refs}} \textendash{} list of output indices

\item[{Returns}] \leavevmode
tuple with outputs buffer and value reference list

\end{description}\end{quote}

\end{fulllineitems}



\chapter{Indices and tables}
\label{\detokenize{index:indices-and-tables}}\begin{itemize}
\item {} 
\DUrole{xref,std,std-ref}{genindex}

\item {} 
\DUrole{xref,std,std-ref}{modindex}

\item {} 
\DUrole{xref,std,std-ref}{search}

\end{itemize}


\renewcommand{\indexname}{Python Module Index}
\begin{sphinxtheindex}
\let\bigletter\sphinxstyleindexlettergroup
\bigletter{f}
\item\relax\sphinxstyleindexentry{files.blueprints.fmu}\sphinxstyleindexpageref{blueprints:\detokenize{module-files.blueprints.fmu}}
\indexspace
\bigletter{m}
\item\relax\sphinxstyleindexentry{main}\sphinxstyleindexpageref{main:\detokenize{module-main}}
\indexspace
\bigletter{s}
\item\relax\sphinxstyleindexentry{src}\sphinxstyleindexpageref{src:\detokenize{module-src}}
\item\relax\sphinxstyleindexentry{src.blueprints.views}\sphinxstyleindexpageref{src.blueprints:\detokenize{module-src.blueprints.views}}
\item\relax\sphinxstyleindexentry{src.clients.models}\sphinxstyleindexpageref{src.clients:\detokenize{module-src.clients.models}}
\item\relax\sphinxstyleindexentry{src.clients.views}\sphinxstyleindexpageref{src.clients:\detokenize{module-src.clients.views}}
\item\relax\sphinxstyleindexentry{src.datasources.models}\sphinxstyleindexpageref{src.datasources:\detokenize{module-src.datasources.models}}
\item\relax\sphinxstyleindexentry{src.datasources.views}\sphinxstyleindexpageref{src.datasources:\detokenize{module-src.datasources.views}}
\item\relax\sphinxstyleindexentry{src.fmus.views}\sphinxstyleindexpageref{src.fmus:\detokenize{module-src.fmus.views}}
\item\relax\sphinxstyleindexentry{src.kafka}\sphinxstyleindexpageref{src:\detokenize{module-src.kafka}}
\item\relax\sphinxstyleindexentry{src.processors.models}\sphinxstyleindexpageref{src.processors:\detokenize{module-src.processors.models}}
\item\relax\sphinxstyleindexentry{src.processors.views}\sphinxstyleindexpageref{src.processors:\detokenize{module-src.processors.views}}
\item\relax\sphinxstyleindexentry{src.server}\sphinxstyleindexpageref{src:\detokenize{module-src.server}}
\item\relax\sphinxstyleindexentry{src.utils}\sphinxstyleindexpageref{src:\detokenize{module-src.utils}}
\item\relax\sphinxstyleindexentry{src.views}\sphinxstyleindexpageref{src:\detokenize{module-src.views}}
\end{sphinxtheindex}

\renewcommand{\indexname}{Index}
\printindex
\end{document}